\chapter{Marco Teórico}

El marco teórico es una sección fundamental que proporciona la base conceptual y teórica sobre la cual se sustenta la investigación. En esta sección, se revisan y analizan las teorías, conceptos, estudios previos y trabajos relevantes que están directamente relacionados con el problema de investigación. El propósito del marco teórico es situar la investigación dentro del contexto académico y científico existente, y justificar el enfoque y los métodos seleccionados.

El marco teórico  debe proporcionar una base sólida y bien fundamentada que respalde la investigación. Al revisar la literatura existente, describir teorías y conceptos clave, y justificar el enfoque metodológico, este marco ayuda a situar la investigación dentro del contexto académico y científico adecuado.

Los componentes del Marco Teórico son:

\begin{enumerate}
    \item Introducción del Marco Teórico:
    \begin{enumerate}
        \item Breve introducción que explique el propósito del marco teórico y su relevancia para la investigación.
        \item Descripción del enfoque general y la estructura del marco teórico.
    \end{enumerate}
    \item Revisión de la Literatura:
    \begin{enumerate}
        \item Antecedentes Históricos: Contexto histórico del tema de estudio, incluyendo el desarrollo y la evolución de las teorías relevantes.
        \item Estudios Previos: Resumen y análisis de investigaciones previas relacionadas con el problema de investigación. Se destacan los hallazgos clave, metodologías utilizadas y brechas identificadas en el conocimiento.
        \item Estado del Arte: Exposición de las tecnologías, métodos y avances más recientes en el área de estudio. Esto puede incluir una revisión de las innovaciones tecnológicas, nuevas metodologías o enfoques teóricos.
    \end{enumerate}
    \item Teorías y Conceptos Relevantes:
    \begin{enumerate}
        \item Teorías Fundamentales: Descripción detallada de las teorías y modelos teóricos que forman la base de la investigación. Se debe explicar cómo estas teorías se aplican al problema de estudio.
        \item Conceptos Clave: Definición y explicación de los conceptos fundamentales utilizados en la investigación. Esto puede incluir términos técnicos, principios de ingeniería y conceptos científicos.
    \end{enumerate}
    \item Modelos y Enfoques Metodológicos:
    \begin{enumerate}
        \item Modelos Matemáticos y Computacionales: Presentación de los modelos teóricos y computacionales que serán utilizados o desarrollados en la investigación. Se deben explicar las ecuaciones y su relevancia.
        \item Metodologías de Investigación: Descripción de las metodologías y técnicas empleadas en estudios anteriores que son relevantes para el enfoque de la investigación actual.
    \end{enumerate}
    \item Relación del Marco Teórico con la Investigación:
    \begin{enumerate}
        \item Justificación del Enfoque: Explicación de cómo las teorías y conceptos revisados justifican y apoyan el enfoque metodológico elegido para la investigación.
        \item Integración de la Literatura: Síntesis de cómo los estudios previos, teorías y conceptos se integran para formar el marco conceptual de la investigación.
    \end{enumerate}
    \item Identificación de Brechas y Oportunidades:
    \begin{enumerate}
        \item Brechas en el Conocimiento: Identificación de las áreas que no han sido suficientemente exploradas o que presentan controversias en la literatura existente.
        \item Oportunidades de Investigación: Presentación de las oportunidades que estas brechas ofrecen para el desarrollo de nuevas investigaciones y contribuciones al campo de estudio.
    \end{enumerate}
\end{enumerate}
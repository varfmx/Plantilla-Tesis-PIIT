Este trabajo que aquí se presenta es el resultado de una etapa de formación académica en la Maestría en Ingeniería para la Innovación Tecnológica, tesis que no habría sido posible sin el apoyo de distintas personas e instituciones, a las cuales les expreso mi más sincero agradecimiento:

\vspace{0.5cm}

A mi director de tesis, el Dr. Viktor Iván Rodríguez Abdalá, quien es mi modelo a seguir por su vasto conocimiento en el área de Telecomunicaciones, por confiar en mí y por siempre tenderme una mano cuando lo necesitaba.

\vspace{0.5cm}

De forma especial y sincera a los miembros del comité de tesis: Dr. Viktor Iván Rodríguez Abdalá, Dr. Héctor Alonso Guerrero Osuna, Dr. Salvador Ibarra Delgado, Dr. Remberto Sandoval Aréchiga y Dr. José Ricardo Gómez Rodríguez, por dedicarme su tiempo, comentarios, ideas, revisiones, propuestas y sugerencias para la realización de este trabajo. Reciban mi admiración y respeto como académicos, así como un profundo agradecimiento por su apoyo personal e incondicional.

\vspace{0.5cm}

A mis compañeros de maestría, quienes con su apoyo, ideas y colaboración hicieron de este camino una experiencia enriquecedora, ayudándome a resolver dudas y compartiendo materiales que facilitaron el aprendizaje en las distintas asignaturas.

\vspace{0.5cm}

Al Posgrado en Ingeniería para la Innovación Tecnológica, que cuenta con un excelente plan de estudios y áreas de especialización, enfocado a la investigación académica y en soluciones prácticas para la industria. A sus docentes, quienes se mantienen siempre a la vanguardia en los temas que conforman cada materia y que destacan como excelentes profesionales al faciliar la transmisión del conocimiento.

\vspace{0.5cm}

Llegar a estos resultados habría sido imposible sin el apoyo de la Universidad Autónoma de Zacatecas y, en especial de la Unidad Académica de Ingeniería Eléctica, que me brindó una educación de calidad con excelentes catedráticos desde el primer momento que en que comencé mis estudios de posgrado.

\vspace{0.5cm}

Al Sistema Nacional de Posgrados (SNP) del Consejo Nacional de Humanidades, Ciencia y Tecnología (CONAHCYT) por su apoyo económico a través de la convocatoria ``Becas Nacionales (Tradicional) 2022-2024`` el cual me proporcionó los medios financieros necesarios para realizar mis estudios de maestría.

\vspace{0.5cm}

Finalmente, este trabajo no habría sido posible sin la constante motivación de familiares y amigos, quienes me alentaron a mantener el empeño en este proyecto. A mis padres, por inculcarme valores, enseñarme la importancia del esfuerzo y estar siempre a mi lado. A mi familia, mi esposa Andrea Liliana, y mis hijos Alejandro, Regina y Luis, por su apoyo fundamental en todo momento.
\chapter{Introducción}

Las comunicaciones móviles son omnipresentes en nuestra vida diaria. En la actualidad prácticamente cualquier dispositivo tiene la habilidad de enviar y recibir datos de manera inalámbrica, ya sea una computadora, dispositivos móviles (como tabletas o teléfonos celulares), vehículos o incluso electrodomésticos. Además existe una categoría de pequeños dispositivos remotos equipados con sensores para monitorear diversos parámetros físicos, los cuales también dependen de las comunicaciones inalámbricas. Estos últimos dispositivos forman parte del Internet de las Cosas o IoT (siglas en inglés de \textit{Internet of Things}).

\vspace{0.5cm}

Los dispositivos IoT conectados a sensores generalmente operan bajo restricciones de consumo de energía ya que suelen estar alimentados por baterías. Además, estos dispositivos típicamente transmiten pequeñas cantidades de datos a intervalos regulares por lo que requieren una tasa de bits muy baja para su comunicación. Por esta razón, la selección adecuada de la tecnología inalámbrica es esencial para garantizar el rendimiento óptimo de los dispositivos IoT.\par

\vspace{0.5cm}

Un factor diferenciador entre dispositivos IoT con características de hardware similares radica principalmente en la implementación del software que lo conforma. Un elemento clave en este aspecto es el uso de un sistema operativo en tiempo real (RTOS, siglas en inglés de \textit{Real Time Operating System}). La incorporación de un RTOS no solo mejora la estabilidad del dispositivo, sino que también permite una administración eficiente de múltiples tareas y una respuesta rápida a eventos en tiempo real.

\vspace{0.5cm}

Un nodo de comunicaciones es un elemento básico en una red IoT, ya que es el dispositivo físico el que recopila información de su entorno, lo procesa y lo envía a una plataforma central o a otros dispositivos. Un nodo de comunicación LoRaWAN (siglas en inglés de \textit{Long Range Wide Area Network})  es un dispositivo que utiliza la tecnología LoRa (siglas en inglés de \textit{Long Range}) para transmitir y recibir datos a largas distancias de manera eficiente con bajo consumo de energía.

\vspace{0.5cm}

Este proyecto consiste en el desarrollo de un nodo de comunicación LoRaWAN  de bajo consumo, utilizando el microcontrolador ESP32 y el módulo de radio LoRa RFM9x en el sistema operativo en tiempo real FreeRTOS, basado en el framework ESP-IDF (siglas en inglés de \textit{Espressif IoT Development Framework}) para su implementación en TTN (siglas en inglés de \textit{The Things Network}).

\section{Planteamiento del problema}

La tecnología LoRaWAN ha demostrado ser altamente eficiente en la creación de redes de largo alcance y de bajo consumo de energía, su uso en combinación con RTOS permite la gestión de tareas en tiempo real donde se requiere priorizar eventos críticos para el manejo de prioridades, concurrencia y comunicación entre tareas.

\vspace{0.5cm}

En este sentido surgen algunas preguntas para implementar el módulo de radio LoRa RFM9x en un microcontrolador ESP32 usando FreeRTOS: ¿Qué se necesita para usar el módulo de radio RFM9x en un microcontrolador ESP32 con FreeRTOS? Si no hay soporte nativo del módulo de radio RFM9x para el ESP-IDF, ¿qué se requiere hacer?

\section{Hipótesis}

El desarrollo de un driver para el módulo de radio RFM9x, compatible con ESP-IDF y FreeRTOS, permitirá establecer una comunicación inalámbrica de bajo consumo y largo alcance en nodos ESP32 para la red TTN.

\section{Justificación}

El creciente desarrollo en redes de comunicación IoT ha impulsado la necesidad de tener soluciones eficientes y de bajo costo que permitan la transmisión de datos a largas distancias con un consumo de energía mínimo. Lo anterior supone un gran desafío, sin embargo, el uso de tecnologías como LoRaWAN permiten lograrlo con una infraestructura simple de telecomunicaciones.

\vspace{0.5cm}

El desarrollo de nodos de comunicación eficientes y confiables que integren hardware y software optimizados para esta tarea sigue siendo un reto. Aunque ya existen soluciones que implementen un RTOS en un nodo LoRaWAN, el uso del módulo RFM9x que se encuentra ampliamente disponible permite construir nodos IoT con dispositivos que están al alcance de cualquier persona. 

\section{Objetivos}

\subsection{Objetivo general}

Desarrollar un nodo de comunicación LoRaWAN basado en el microcontrolador ESP32 y el módulo de radio LoRa RFM9x que junto con el sistema operativo en tiempo real FreeRTOS permita su integración en una red TTN.


\subsection{Objetivos particulares}

\begin{itemize}
   \item Desarrollar una biblioteca de funciones para el manejo del módulo RFM9x para su uso en el entorno ESP-IDF, incluyendo funciones para configuración, transmisión y recepción de datos.
   \item Desarrollar un prototipo de laboratorio para una comunicación punto a punto y su integración a la red TTN.
\end{itemize}


\section{Alcances y limitaciones}

\subsection{Alcances}

\begin{itemize}
   \item Desarrollar el nodo LoRaWAN utilizando el microcontrolador ESP32 y el sistema operativo FreeRTOS.
\end{itemize}


\subsection{Limitaciones}

\begin{itemize}
   \item El proyecto se centra en el desarrollo de un nodo LoRaWAN, no en la implementación de una red LoRaWAN completa.
   \item No se incluye el desarrollo de aplicaciones de alto nivel para procesamiento de datos.
   \item El proyecto se limita al uso del microcontrolador ESP32 y el módulo de radio RFM9x. No se consideran otras opciones de hardware.
   \item Se utiliza el sistema operativo FreeRTOS, no se exploran otras opciones de RTOS.
    \item Las pruebas se realizarán en un entorno controlado, sin considerar interferencias externas o condiciones ambientales extremas.
\end{itemize}


\section{Metodología}

Procedimiento para el desarrollo del driver para el RFM9x:

\begin{enumerate}
    \item Familiarización con el entorno de desarrollo ESP-IDF. \\
    Iniciar con la instalación y configuración del entorno de desarrollo ESP-IDF para el microcontrolador ESP32, asegurando el conocimiento básico de sus herramientas y componentes.
    \item Comprensión del protocolo de comunicación SPI. \\
    Estudiar la implementación de la comunicación mediante el protocolo SPI (siglas en inglés de Serial Peripheral Interface) en el ESP32, ya que será el protocolo de comunicación con el módulo RFM9x.
    \item Análisis del driver disponible en otras plataformas de desarollo sobre el RFM9x. \\
    Revisar y analizar el código fuente del driver para el módulo RFM9x disponible otras plataformas, proporcionado por Espressif. Este código servirá como referencia para comprender los elementos básicos de la comunicación con el módulo de radio.
    \item Estudio de la hoja de datos del RFM9x. \\
    Examinar la hoja de datos del RFM9x, enfocándose en escribir registros a través de SPI, y en los parámetros específicos necesarios para la configuración e inicialización del módulo.
    \item Desarrollo del código básico de lectura y escritura de registros. \\
    Implementar código en ESP-IDF que permita la lectura y escritura de registros en el RFM9x utilizando SPI.
    \item Análisis de la transmisión de datos. \\
    Estudiar el capítulo de transmisión de datos en la hoja de datos del RFM9x para comprender los registros implicados en el proceso de envío y recepción de datos.
    \item Desarrollo del pseudocódigo para la transmisión. \\
    Crear un pseudocódigo que detalle el flujo de la transmisión de datos, identificando los registros de lectura y escritura para lograr una transmisión exitosa.
    \item Desarrollo del código de transmisión y recepción. \\
    Implementar el código en C utilizando ESP-IDF para la transmisión y recepción de datos a través del módulo de radio RFM9x.
    \item Verificación y pruebas del código. \\
    Desarrollar un proceso de lectura continua de registros para asegurar que los datos se escriben correctamente en el registro correspondiente, evitando errores de comunicación.
    \item Iteración y optimización del código. \\
    Realizar pruebas iterativas para mejorar y optimizar el código del driver. Cada ciclo de prueba proporcionará retroalimentación para corregir errores y mejorar el rendimiento, hasta obtener un driver funcional y eficiente.
    \item Prototipo de laboratorio. \\
    Desarrollar un prototipo de laboratorio para probar la comunicación punto a punto entre dos nodos usando LoRa.
    \item Implementación del stack LoRaWAN (TTN). \\
    Una vez funcional el driver, se procederá a la integración con el stack LoRaWAN mediante la librería ttn-esp32, para implementar un nodo LoRaWAN que pueda conectarse a la red TTN.
\end{enumerate}


% ==================================================================
% INICIA PREÁMBULO DEL DOCUMENTO
% ==================================================================

% -----------DECLARACIÓN DE TIPO DE DOCUMENTO A DISEÑAR-------------

\documentclass[12pt]{uaztesis}

% % --------------DECLARACIÓN DE PAQUETES A UTILIZAR------------------

\usepackage[spanish,mexico]{babel}
\usepackage{subfigure}
\usepackage{multirow,rotating}
\usepackage[hidelinks]{hyperref}
\usepackage{graphicx}
\usepackage{float}
\usepackage{fancyhdr}
\usepackage{listings}
\usepackage{amssymb,amsmath,ams fonts}
\usepackage{tabularx,colortbl,xspace,rotating,booktabs,longtable,multirow}
\usepackage{soulutf8}
\usepackage{multirow} % para las tablas
\usepackage{longtable} % para tablas largas
\usepackage{multirow, array} % para las tablas
\usepackage{float} % para usar [H]
\usepackage{color,curves}
\usepackage{chngcntr}
\usepackage{csquotes}
\usepackage{minted}
\renewcommand\listingscaption{C\'odigo}
\numberwithin{listing}{chapter}

% % --DECLARACIÓN DE RUTA DE CARPETA DONDE SE ENCUENTRAN LAS FIGURAS--
\graphicspath{{Imagenes/}}
\psfull

% % --------------DECLARACIÓN DEL ESTILO DE PÁGINA--------------------
\pagestyle{thesis}
\noappendixtables
\noappendixfigures

% % ===================================================================
% % INICIA EL CONTENIDO DEL DOCUMENTO
% % ===================================================================

\begin{document}

%\thesism % Tesis de maestria
\thesisd % Tesis de doctorado
%\thesis % Tesis licenciatura
%\degree{Ingeniero en Robótica y Mecatrónica} % Solo para licenciatura

% % Introducción en sílabas de palabras desconocidas por LaTeX
 \hyphenation{}

% % Declaración de número de páginas en números Romanos
 \clearpage\pagenumbering{roman}

% % ------------------INTRODUCCIÓN DE DATOS----------------------------

% % Título de la tesis, autor, y grado a recibir:
\title{Título de la tesis} 
\author{Estudiante}

% % Grado y nombre de asesores de tesis:
\advisortitle{Grado} \advisorname{Asesor} 
\gradosegundoasesor{Grado} \nombresegundoasesor{Asesor}
\gradotercerasesor{Grado} \nombretercerasesor{Asesor}

\date{\the\year{}}

% % ---------GENERACIÓN DE PÁGINAS PRELIMINARES DEL DOCUMENTO----------

% % Genera página de presentación
 \maketitle

% % Genera páginas de resumen
\begin{resumen}
  Un resumen de tesis es una síntesis breve y concisa de los elementos clave de una tesis. Su propósito es proporcionar una visión general del trabajo, permitiendo al lector comprender rápidamente el objetivo, la metodología, los hallazgos principales y las conclusiones del estudio. 

Un buen resumen debe captar la esencia de la investigación y destacar sus puntos más importantes, debe ser completo en cuanto al contenido del trabajo, comprensible, sencillo y preciso, no debe contener citas bibliográficas, figuras o tablas. 

Este debe de ser escrito en un solo párrafo, no superar las 350 palabras en el caso de las tesis de maestría y las 500 en las de doctorado.          
\end{resumen}

% % Genera página de dedicatoria (opcional)
\begin{dedicatoria}
  En esta sección se puede expresar su gratitud y reconocimiento a personas o entidades que han influido positivamente en su vida académica, profesional o personal, y que han brindado apoyo durante el proceso de investigación y redacción de la tesis. 

Es una parte personal y emotiva del documento, que permite al autor reconocer y honrar a quienes contribuyeron a su éxito y esfuerzo académico. Esta debe ser corta, en una sola página.          
\end{dedicatoria}

 % Genera página de agradecimientos (opcional)
\begin{agradecimientos}
  En esta sección se expresa su gratitud y reconocimiento a todas las personas e instituciones que han contribuido directa o indirectamente al desarrollo y culminación de su trabajo de investigación. Esta sección es una oportunidad para mostrar aprecio por el apoyo, la orientación y los recursos proporcionados durante el proceso de elaboración de la tesis.

La estructura es la siguiente:

\begin{enumerate}
    \item Al director de tesis
    \item A familiares y amigos
    \item A los compañeros de investigación
    \item Al Posgrado en Ingeniería para la Innovación Tecnológica y docentes
    \item A la Universidad Autónoma de Zacatecas
    \item Al CONAHCYT (en caso de ser becario)
\end{enumerate}

Un ejemplo de agradecimiento al CONAHCYT es: Al Sistema Nacional de Posgrados (SNP) del Consejo Nacional de Humanidades, Ciencia y Tecnología (CONAHCYT) por su apoyo económico a través de la convocatoria "Becas Nacionales (Tradicional) 20xx-20xx".        
\end{agradecimientos}

% % Genera páginas del contenido general y lista de figuras y tablas
\tableofcontents
\listoffigures                 
\listoftables                  

\begin{nomenclatura}
    \begin{description}
\item{\makebox[2.5cm][l]{$cd$}} Corriente directa.
\item{\makebox[2.5cm][l]{$ca$}} Corriente alterna.
\item{\makebox[2.5cm][l]{$NdFeB$}} Neudimio-Fierro-Boro.
\item{\makebox[2.5cm][l]{$FEM$}} Fuerza electromotriz.
\item{\makebox[2.5cm][l]{$FMM$}} Fuerza magnetomotriz.
\end{description}
\end{nomenclatura}

\begin{listofsymbols}
    \begin{description}
\item{\makebox[1.2cm][l]{$\mu_0$} \makebox[2.2cm][l] {$[Wb/A \cdot m]$}} Permeabilidad magnética del espacio libre.
\item{\makebox[1.2cm][l]{$\mu_{rFe}$} \makebox[2.2cm][l] { }} Permeabilidad magnética relativa del acero.
\item{\makebox[1.2cm][l]{$\rho$} \makebox[2.2cm][l] {$[kg/m^{3}]$}} Densidad de masa del aire.
\item{\makebox[1.2cm][l]{$\rho_{cu}$} \makebox[2.2cm][l] {$[kg/m^{3}]$}} Densidad de masa del cobre.
\item{\makebox[1.2cm][l]{$H$} \makebox[2.2cm][l] {$[A/m]$}} Intensidad de campo magnético.
\item{\makebox[1.2cm][l]{$\phi_r$} \makebox[2.2cm][l] {$[Wb]$}}  Flujo magnético remanente.
\item{\makebox[1.2cm][l]{$\eta$} \makebox[2.2cm][l] { }} Eficiencia.
\item{\makebox[1.2cm][l]{$\lambda$} \makebox[2.2cm][l] { }} Velocidad específica o TSR.
\item{\makebox[1.2cm][l]{$T$} \makebox[2.2cm][l] {$[N \cdot m]$}} Par o torque.
\item{\makebox[1.2cm][l]{$I$} \makebox[2.2cm][l] {$[A]$}} Corriente eléctrica.
\item{\makebox[1.2cm][l]{$R$} \makebox[2.2cm][l] {$[\Omega]$}} Resistencia eléctrica.
\end{description}
\end{listofsymbols}

\begin{glosario}
   \input{06_Glosario}
\end{glosario}

% %--------INCLUSIÓN DE LOS ARCHIVOS (CAPÍTULOS) DEL DOCUMENTO--------

% % Declaración de número de páginas en números Arábigos
 \clearpage\pagenumbering{arabic}

% incluye los captitulos de la tesis
\chapter{Introducción}

Las comunicaciones móviles son omnipresentes en nuestra vida diaria. En la actualidad prácticamente cualquier dispositivo tiene la habilidad de enviar y recibir datos de manera inalámbrica, ya sea una computadora, dispositivos móviles (como tabletas o teléfonos celulares), vehículos o incluso electrodomésticos. Además existe una categoría de pequeños dispositivos remotos equipados con sensores para monitorear diversos parámetros físicos, los cuales también dependen de las comunicaciones inalámbricas. Estos últimos dispositivos forman parte del Internet de las Cosas o IoT (siglas en inglés de \textit{Internet of Things}).

\vspace{0.5cm}

Los dispositivos IoT conectados a sensores generalmente operan bajo restricciones de consumo de energía ya que suelen estar alimentados por baterías. Además, estos dispositivos típicamente transmiten pequeñas cantidades de datos a intervalos regulares por lo que requieren una tasa de bits muy baja para su comunicación. Por esta razón, la selección adecuada de la tecnología inalámbrica es esencial para garantizar el rendimiento óptimo de los dispositivos IoT.\par

\vspace{0.5cm}

Un factor diferenciador entre dispositivos IoT con características de hardware similares radica principalmente en la implementación del software que lo conforma. Un elemento clave en este aspecto es el uso de un sistema operativo en tiempo real (RTOS, siglas en inglés de \textit{Real Time Operating System}). La incorporación de un RTOS no solo mejora la estabilidad del dispositivo, sino que también permite una administración eficiente de múltiples tareas y una respuesta rápida a eventos en tiempo real.

\vspace{0.5cm}

Un nodo de comunicaciones es un elemento básico en una red IoT, ya que es el dispositivo físico el que recopila información de su entorno, lo procesa y lo envía a una plataforma central o a otros dispositivos. Un nodo de comunicación LoRaWAN (siglas en inglés de \textit{Long Range Wide Area Network})  es un dispositivo que utiliza la tecnología LoRa (siglas en inglés de \textit{Long Range}) para transmitir y recibir datos a largas distancias de manera eficiente con bajo consumo de energía.

\vspace{0.5cm}

Este proyecto consiste en el desarrollo de un nodo de comunicación LoRaWAN  de bajo consumo, utilizando el microcontrolador ESP32 y el módulo de radio LoRa RFM9x en el sistema operativo en tiempo real FreeRTOS, basado en el framework ESP-IDF (siglas en inglés de \textit{Espressif IoT Development Framework}) para su implementación en TTN (siglas en inglés de \textit{The Things Network}).

\section{Planteamiento del problema}

La tecnología LoRaWAN ha demostrado ser altamente eficiente en la creación de redes de largo alcance y de bajo consumo de energía, su uso en combinación con RTOS permite la gestión de tareas en tiempo real donde se requiere priorizar eventos críticos para el manejo de prioridades, concurrencia y comunicación entre tareas.

\vspace{0.5cm}

En este sentido surgen algunas preguntas para implementar el módulo de radio LoRa RFM9x en un microcontrolador ESP32 usando FreeRTOS: ¿Qué se necesita para usar el módulo de radio RFM9x en un microcontrolador ESP32 con FreeRTOS? Si no hay soporte nativo del módulo de radio RFM9x para el ESP-IDF, ¿qué se requiere hacer?

\section{Hipótesis}

El desarrollo de un driver para el módulo de radio RFM9x, compatible con ESP-IDF y FreeRTOS, permitirá establecer una comunicación inalámbrica de bajo consumo y largo alcance en nodos ESP32 para la red TTN.

\section{Justificación}

El creciente desarrollo en redes de comunicación IoT ha impulsado la necesidad de tener soluciones eficientes y de bajo costo que permitan la transmisión de datos a largas distancias con un consumo de energía mínimo. Lo anterior supone un gran desafío, sin embargo, el uso de tecnologías como LoRaWAN permiten lograrlo con una infraestructura simple de telecomunicaciones.

\vspace{0.5cm}

El desarrollo de nodos de comunicación eficientes y confiables que integren hardware y software optimizados para esta tarea sigue siendo un reto. Aunque ya existen soluciones que implementen un RTOS en un nodo LoRaWAN, el uso del módulo RFM9x que se encuentra ampliamente disponible permite construir nodos IoT con dispositivos que están al alcance de cualquier persona. 

\section{Objetivos}

\subsection{Objetivo general}

Desarrollar un nodo de comunicación LoRaWAN basado en el microcontrolador ESP32 y el módulo de radio LoRa RFM9x que junto con el sistema operativo en tiempo real FreeRTOS permita su integración en una red TTN.


\subsection{Objetivos particulares}

\begin{itemize}
   \item Desarrollar una biblioteca de funciones para el manejo del módulo RFM9x para su uso en el entorno ESP-IDF, incluyendo funciones para configuración, transmisión y recepción de datos.
   \item Desarrollar un prototipo de laboratorio para una comunicación punto a punto y su integración a la red TTN.
\end{itemize}


\section{Alcances y limitaciones}

\subsection{Alcances}

\begin{itemize}
   \item Desarrollar el nodo LoRaWAN utilizando el microcontrolador ESP32 y el sistema operativo FreeRTOS.
\end{itemize}


\subsection{Limitaciones}

\begin{itemize}
   \item El proyecto se centra en el desarrollo de un nodo LoRaWAN, no en la implementación de una red LoRaWAN completa.
   \item No se incluye el desarrollo de aplicaciones de alto nivel para procesamiento de datos.
   \item El proyecto se limita al uso del microcontrolador ESP32 y el módulo de radio RFM9x. No se consideran otras opciones de hardware.
   \item Se utiliza el sistema operativo FreeRTOS, no se exploran otras opciones de RTOS.
    \item Las pruebas se realizarán en un entorno controlado, sin considerar interferencias externas o condiciones ambientales extremas.
\end{itemize}


\section{Metodología}

Procedimiento para el desarrollo del driver para el RFM9x:

\begin{enumerate}
    \item Familiarización con el entorno de desarrollo ESP-IDF. \\
    Iniciar con la instalación y configuración del entorno de desarrollo ESP-IDF para el microcontrolador ESP32, asegurando el conocimiento básico de sus herramientas y componentes.
    \item Comprensión del protocolo de comunicación SPI. \\
    Estudiar la implementación de la comunicación mediante el protocolo SPI (siglas en inglés de Serial Peripheral Interface) en el ESP32, ya que será el protocolo de comunicación con el módulo RFM9x.
    \item Análisis del driver disponible en otras plataformas de desarollo sobre el RFM9x. \\
    Revisar y analizar el código fuente del driver para el módulo RFM9x disponible otras plataformas, proporcionado por Espressif. Este código servirá como referencia para comprender los elementos básicos de la comunicación con el módulo de radio.
    \item Estudio de la hoja de datos del RFM9x. \\
    Examinar la hoja de datos del RFM9x, enfocándose en escribir registros a través de SPI, y en los parámetros específicos necesarios para la configuración e inicialización del módulo.
    \item Desarrollo del código básico de lectura y escritura de registros. \\
    Implementar código en ESP-IDF que permita la lectura y escritura de registros en el RFM9x utilizando SPI.
    \item Análisis de la transmisión de datos. \\
    Estudiar el capítulo de transmisión de datos en la hoja de datos del RFM9x para comprender los registros implicados en el proceso de envío y recepción de datos.
    \item Desarrollo del pseudocódigo para la transmisión. \\
    Crear un pseudocódigo que detalle el flujo de la transmisión de datos, identificando los registros de lectura y escritura para lograr una transmisión exitosa.
    \item Desarrollo del código de transmisión y recepción. \\
    Implementar el código en C utilizando ESP-IDF para la transmisión y recepción de datos a través del módulo de radio RFM9x.
    \item Verificación y pruebas del código. \\
    Desarrollar un proceso de lectura continua de registros para asegurar que los datos se escriben correctamente en el registro correspondiente, evitando errores de comunicación.
    \item Iteración y optimización del código. \\
    Realizar pruebas iterativas para mejorar y optimizar el código del driver. Cada ciclo de prueba proporcionará retroalimentación para corregir errores y mejorar el rendimiento, hasta obtener un driver funcional y eficiente.
    \item Prototipo de laboratorio. \\
    Desarrollar un prototipo de laboratorio para probar la comunicación punto a punto entre dos nodos usando LoRa.
    \item Implementación del stack LoRaWAN (TTN). \\
    Una vez funcional el driver, se procederá a la integración con el stack LoRaWAN mediante la librería ttn-esp32, para implementar un nodo LoRaWAN que pueda conectarse a la red TTN.
\end{enumerate}


\chapter{Marco Teórico}

El marco teórico es una sección fundamental que proporciona la base conceptual y teórica sobre la cual se sustenta la investigación. En esta sección, se revisan y analizan las teorías, conceptos, estudios previos y trabajos relevantes que están directamente relacionados con el problema de investigación. El propósito del marco teórico es situar la investigación dentro del contexto académico y científico existente, y justificar el enfoque y los métodos seleccionados.

El marco teórico  debe proporcionar una base sólida y bien fundamentada que respalde la investigación. Al revisar la literatura existente, describir teorías y conceptos clave, y justificar el enfoque metodológico, este marco ayuda a situar la investigación dentro del contexto académico y científico adecuado.

Los componentes del Marco Teórico son:

\begin{enumerate}
    \item Introducción del Marco Teórico:
    \begin{enumerate}
        \item Breve introducción que explique el propósito del marco teórico y su relevancia para la investigación.
        \item Descripción del enfoque general y la estructura del marco teórico.
    \end{enumerate}
    \item Revisión de la Literatura:
    \begin{enumerate}
        \item Antecedentes Históricos: Contexto histórico del tema de estudio, incluyendo el desarrollo y la evolución de las teorías relevantes.
        \item Estudios Previos: Resumen y análisis de investigaciones previas relacionadas con el problema de investigación. Se destacan los hallazgos clave, metodologías utilizadas y brechas identificadas en el conocimiento.
        \item Estado del Arte: Exposición de las tecnologías, métodos y avances más recientes en el área de estudio. Esto puede incluir una revisión de las innovaciones tecnológicas, nuevas metodologías o enfoques teóricos.
    \end{enumerate}
    \item Teorías y Conceptos Relevantes:
    \begin{enumerate}
        \item Teorías Fundamentales: Descripción detallada de las teorías y modelos teóricos que forman la base de la investigación. Se debe explicar cómo estas teorías se aplican al problema de estudio.
        \item Conceptos Clave: Definición y explicación de los conceptos fundamentales utilizados en la investigación. Esto puede incluir términos técnicos, principios de ingeniería y conceptos científicos.
    \end{enumerate}
    \item Modelos y Enfoques Metodológicos:
    \begin{enumerate}
        \item Modelos Matemáticos y Computacionales: Presentación de los modelos teóricos y computacionales que serán utilizados o desarrollados en la investigación. Se deben explicar las ecuaciones y su relevancia.
        \item Metodologías de Investigación: Descripción de las metodologías y técnicas empleadas en estudios anteriores que son relevantes para el enfoque de la investigación actual.
    \end{enumerate}
    \item Relación del Marco Teórico con la Investigación:
    \begin{enumerate}
        \item Justificación del Enfoque: Explicación de cómo las teorías y conceptos revisados justifican y apoyan el enfoque metodológico elegido para la investigación.
        \item Integración de la Literatura: Síntesis de cómo los estudios previos, teorías y conceptos se integran para formar el marco conceptual de la investigación.
    \end{enumerate}
    \item Identificación de Brechas y Oportunidades:
    \begin{enumerate}
        \item Brechas en el Conocimiento: Identificación de las áreas que no han sido suficientemente exploradas o que presentan controversias en la literatura existente.
        \item Oportunidades de Investigación: Presentación de las oportunidades que estas brechas ofrecen para el desarrollo de nuevas investigaciones y contribuciones al campo de estudio.
    \end{enumerate}
\end{enumerate}
\chapter{Desarrollo}

\textbf{Cómo citar utilizando BibLaTeX}

BibLaTeX es una herramienta  para gestionar y generar referencias en documentos escritos con LaTeX. Ofrece una variedad de comandos para citar fuentes de manera adecuada según el contexto.

A continuación, se describe cómo usar los principales comandos para citar y se incluyen ejemplos prácticos.

\textbf{Comandos principales para citar}

\begin{enumerate}
    \item \textbackslash cite
    Este comando genera una cita compacta que usualmente incluye solo el autor y el año, o un número si se utiliza un estilo numérico.

    Uso típico: Se emplea en notas al pie o en listados donde no es necesario integrar la cita en el texto.
    
    Ejemplo:
    
    
    Según los resultados presentados en varios estudios \textbackslash cite\{smith2010data\}, el modelo es efectivo.
    
    Según los resultados presentados en varios estudios \cite{smith2010data}, el modelo es efectivo.
    
    \item \textbackslash parencite

    Genera una cita entre paréntesis que incluye el autor y el año o el identificador correspondiente al estilo elegido.
    Uso típico: Ideal para citas parentéticas en texto fluido.

    Ejemplo:

    El rendimiento de los sistemas GNSS ha sido evaluado (véase \textbackslash parencite\{johnson2015gps\}).


    El rendimiento de los sistemas GNSS ha sido evaluado (véase \parencite{johnson2015gps}).

    \item \textbackslash textcite

    Integra la referencia directamente en el flujo del texto, mencionando explícitamente el nombre del autor y el año.
    Uso típico: Perfecto para dar énfasis a los autores en el cuerpo del texto.

    Ejemplo:

    \textbackslash textcite\{doe2018antenna\} destacan que las antenas log-periódicas son altamente efectivas para aplicaciones de radar.

    \textcite{doe2018antenna} destacan que las antenas log-periódicas son altamente efectivas para aplicaciones de radar.

\end{enumerate}

\textbf{Ejemplo práctico de uso en un documento}

El desarrollo de antenas para aplicaciones de radar ha sido un tema de amplio interés. \textbackslash textcite\{smith2010data\} propusieron un diseño optimizado para frecuencias de microondas. Otros estudios, como los de \textbackslash parencite\{johnson2015gps\}, se han enfocado en la integración con sistemas GNSS.

Un análisis comparativo demuestra que \textbackslash cite\{doe2018antenna\} lograron mejores resultados al implementar estructuras log-periódicas.


El desarrollo de antenas para aplicaciones de radar ha sido un tema de amplio interés. \textcite{smith2010data} propusieron un diseño optimizado para frecuencias de microondas. Otros estudios, como los de \parencite{johnson2015gps}, se han enfocado en la integración con sistemas GNSS.

Un análisis comparativo demuestra que \cite{doe2018antenna} lograron mejores resultados al implementar estructuras log-periódicas.

\textbf{Citas con varias fuentes}

Cuando necesitas citar varias fuentes al mismo tiempo con BibLaTeX, puedes incluir múltiples claves de entrada en el mismo comando de citación, separadas por comas. Esto se aplica a los comandos como \textbackslash cite, \textbackslash parencite y \textbackslash textcite. A continuación, se muestra cómo hacerlo con ejemplos prácticos.

Ejemplo de citar varios autores al mismo tiempo

\begin{enumerate}
    \item Con \textbackslash cite

    Genera una cita compacta con todas las fuentes mencionadas.
    
    Diversos estudios han analizado el impacto de las antenas log-periódicas en aplicaciones de radar \textbackslash cite\{smith2010data, johnson2015gps, doe2018antenna\}.

    Diversos estudios han analizado el impacto de las antenas log-periódicas en aplicaciones de radar \cite{smith2010data, johnson2015gps, doe2018antenna}.

    \item Con \textbackslash parencite

    Incluye la lista de citas en un formato parentético.
    
    El rendimiento del sistema ha sido evaluado previamente (véase \textbackslash parencite\{smith2010data, johnson2015gps, doe2018antenna\}).

    El rendimiento del sistema ha sido evaluado previamente (véase \parencite{smith2010data, johnson2015gps, doe2018antenna}).

    \item Con \textbackslash textcite

    Integra varios autores en el texto.
    
    Trabajos recientes como los de \textbackslash textcite\{smith2010data\}, \textbackslash textcite\{johnson2015gps\} y \textbackslash textcite\{doe2018antenna\} han destacado diferentes aspectos del diseño y rendimiento de estas tecnologías.

    Trabajos recientes como los de \textcite{smith2010data}, \textcite{johnson2015gps} y \textcite{doe2018antenna} han destacado diferentes aspectos del diseño y rendimiento de estas tecnologías.
    
\end{enumerate}

\textbf{Resumen de recomendaciones para citar}

\begin{enumerate}
    \item \textbackslash cite: Úsalo para citas generales o en listados compactos.
    \item \textbackslash parencite: Utilízalo cuando necesites incluir la cita entre paréntesis.
    \item \textbackslash textcite: Prefiérelo para integrar la referencia de forma fluida en el texto.
\end{enumerate}

El archivo Referencias.bib es donde se almacenan las entradas bibliográficas.

Esta flexibilidad te permite adaptar tus citas al estilo de redacción que prefieras, manteniendo un formato académico correcto.
\input{Cap4}
\input{Cap5}

% % ------------------INTRODUCCIÓN DE REFERENCIAS---------------------

\nocite{*}
\chapter*{Referencias bibliográficas}     
\addcontentsline{toc}{chapter}{Referencias bibliográficas}
\printbibliography[heading=none]

% % ---------------------INTRODUCCIÓN DE APÉNDICES--------------------

\begin{apendices}
\input{A_apen}
\input{B_apen}
\end{apendices}


\end{document}
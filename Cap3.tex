\chapter{Desarrollo}

\textbf{Cómo citar utilizando BibLaTeX}

BibLaTeX es una herramienta  para gestionar y generar referencias en documentos escritos con LaTeX. Ofrece una variedad de comandos para citar fuentes de manera adecuada según el contexto.

A continuación, se describe cómo usar los principales comandos para citar y se incluyen ejemplos prácticos.

\textbf{Comandos principales para citar}

\begin{enumerate}
    \item \textbackslash cite
    Este comando genera una cita compacta que usualmente incluye solo el autor y el año, o un número si se utiliza un estilo numérico.

    Uso típico: Se emplea en notas al pie o en listados donde no es necesario integrar la cita en el texto.
    
    Ejemplo:
    
    
    Según los resultados presentados en varios estudios \textbackslash cite\{smith2010data\}, el modelo es efectivo.
    
    Según los resultados presentados en varios estudios \cite{smith2010data}, el modelo es efectivo.
    
    \item \textbackslash parencite

    Genera una cita entre paréntesis que incluye el autor y el año o el identificador correspondiente al estilo elegido.
    Uso típico: Ideal para citas parentéticas en texto fluido.

    Ejemplo:

    El rendimiento de los sistemas GNSS ha sido evaluado (véase \textbackslash parencite\{johnson2015gps\}).


    El rendimiento de los sistemas GNSS ha sido evaluado (véase \parencite{johnson2015gps}).

    \item \textbackslash textcite

    Integra la referencia directamente en el flujo del texto, mencionando explícitamente el nombre del autor y el año.
    Uso típico: Perfecto para dar énfasis a los autores en el cuerpo del texto.

    Ejemplo:

    \textbackslash textcite\{doe2018antenna\} destacan que las antenas log-periódicas son altamente efectivas para aplicaciones de radar.

    \textcite{doe2018antenna} destacan que las antenas log-periódicas son altamente efectivas para aplicaciones de radar.

\end{enumerate}

\textbf{Ejemplo práctico de uso en un documento}

El desarrollo de antenas para aplicaciones de radar ha sido un tema de amplio interés. \textbackslash textcite\{smith2010data\} propusieron un diseño optimizado para frecuencias de microondas. Otros estudios, como los de \textbackslash parencite\{johnson2015gps\}, se han enfocado en la integración con sistemas GNSS.

Un análisis comparativo demuestra que \textbackslash cite\{doe2018antenna\} lograron mejores resultados al implementar estructuras log-periódicas.


El desarrollo de antenas para aplicaciones de radar ha sido un tema de amplio interés. \textcite{smith2010data} propusieron un diseño optimizado para frecuencias de microondas. Otros estudios, como los de \parencite{johnson2015gps}, se han enfocado en la integración con sistemas GNSS.

Un análisis comparativo demuestra que \cite{doe2018antenna} lograron mejores resultados al implementar estructuras log-periódicas.

\textbf{Citas con varias fuentes}

Cuando necesitas citar varias fuentes al mismo tiempo con BibLaTeX, puedes incluir múltiples claves de entrada en el mismo comando de citación, separadas por comas. Esto se aplica a los comandos como \textbackslash cite, \textbackslash parencite y \textbackslash textcite. A continuación, se muestra cómo hacerlo con ejemplos prácticos.

Ejemplo de citar varios autores al mismo tiempo

\begin{enumerate}
    \item Con \textbackslash cite

    Genera una cita compacta con todas las fuentes mencionadas.
    
    Diversos estudios han analizado el impacto de las antenas log-periódicas en aplicaciones de radar \textbackslash cite\{smith2010data, johnson2015gps, doe2018antenna\}.

    Diversos estudios han analizado el impacto de las antenas log-periódicas en aplicaciones de radar \cite{smith2010data, johnson2015gps, doe2018antenna}.

    \item Con \textbackslash parencite

    Incluye la lista de citas en un formato parentético.
    
    El rendimiento del sistema ha sido evaluado previamente (véase \textbackslash parencite\{smith2010data, johnson2015gps, doe2018antenna\}).

    El rendimiento del sistema ha sido evaluado previamente (véase \parencite{smith2010data, johnson2015gps, doe2018antenna}).

    \item Con \textbackslash textcite

    Integra varios autores en el texto.
    
    Trabajos recientes como los de \textbackslash textcite\{smith2010data\}, \textbackslash textcite\{johnson2015gps\} y \textbackslash textcite\{doe2018antenna\} han destacado diferentes aspectos del diseño y rendimiento de estas tecnologías.

    Trabajos recientes como los de \textcite{smith2010data}, \textcite{johnson2015gps} y \textcite{doe2018antenna} han destacado diferentes aspectos del diseño y rendimiento de estas tecnologías.
    
\end{enumerate}

\textbf{Resumen de recomendaciones para citar}

\begin{enumerate}
    \item \textbackslash cite: Úsalo para citas generales o en listados compactos.
    \item \textbackslash parencite: Utilízalo cuando necesites incluir la cita entre paréntesis.
    \item \textbackslash textcite: Prefiérelo para integrar la referencia de forma fluida en el texto.
\end{enumerate}

El archivo Referencias.bib es donde se almacenan las entradas bibliográficas.

Esta flexibilidad te permite adaptar tus citas al estilo de redacción que prefieras, manteniendo un formato académico correcto.
Un resumen de tesis es una síntesis breve y concisa de los elementos clave de una tesis. Su propósito es proporcionar una visión general del trabajo, permitiendo al lector comprender rápidamente el objetivo, la metodología, los hallazgos principales y las conclusiones del estudio. 

Un buen resumen debe captar la esencia de la investigación y destacar sus puntos más importantes, debe ser completo en cuanto al contenido del trabajo, comprensible, sencillo y preciso, no debe contener citas bibliográficas, figuras o tablas. 

Este debe de ser escrito en un solo párrafo, no superar las 350 palabras en el caso de las tesis de maestría y las 500 en las de doctorado.